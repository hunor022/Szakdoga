%----------------------------------------------------------------------------
\chapter{Mérőrendszer fejlesztése}
\label{sec:Fejleszt}
%----------------------------------------------------------------------------


A felszedő adapteren alkalmazott elektromos készülékekhez $12$ V tápfeszültség van alkalmazva, így az egyik fő követelménye a fejlesztendő rendszernek, hogy ezen a feszültségen működjön. A korábban már említett por- és olajszennyezés, valamint a kompaktság és modularitás is fontos, hogy szem előtt maradjon. A rendszer kialakításában nagy szerepet játszik még a biztonságtechnikai megfontolás, valamint a meglévő kialakításra való ráillesztés, a támogató mérnöki munka lecsökkentése.

%----------------------------------------------------------------------------
\section{Szenzorok}
%----------------------------------------------------------------------------

A feladat során három szenzor alkalmazása szükséges, melyek mind fordulatszámot mérnek, azonban különböző céllal és kialakításban. Az első cél a nyomatékhatároló két végén való összehasonlítás, amelyhez két fordulatszámmérő szenzor alkalmazása szükséges. Ezen felül a fogakkal ellátott felszedő tengely fordulatszámának mérése is a feladat része, ez azonban a visszajelzésben tájékoztató jellegű, biztonságtechnikai feladata nincs.

%----------------------------------------------------------------------------
\subsection{Szenzorválasztás}
%----------------------------------------------------------------------------

Ezek alapján a szenzornak a következő kritériumoknak kell megfelelnie:

\textbf{Por- és olajállóság}: a szenzor élettartamát nem befolyásolhatják ezek a szennyező anyagok, valamint szennyezett környezetben a mérés pontosságát meg kell tartania. Az első kritériumnak a legtöbb ipari kialakítású szenzor megfelel, hiszen megfelelő ház, burkolat védi őket a külvilágtól. A második kritérium


\subsubsection{Mérési elvek}
miért nem azokat választottam

\subsubsection{Szenzor paraméterei}
követelmények a technológiával
mérési frekvencia
mérési távolság
%----------------------------------------------------------------------------
\subsubsection{Kialakítás}
%----------------------------------------------------------------------------
burkolat
dugalyos
%----------------------------------------------------------------------------
\subsection{Elhelyezés}
%----------------------------------------------------------------------------
A \ref{szenzorok} ábrán látható a tengely két oldalán való szenzor kialakítása. Ezek a szenzorok a takarmány összeterelésére alkalmazott csiga két oldalán helyezkednek el, mivel a nyomatékhatároló felőli oldalon nem állt rendelkezésre felület még egy szenzor felhelyezésére, így a csiga tengelyének a túloldalára erősített lemezen keresztül lesz alkalmunk azt mérni.

kábelezés
3 erezet
tápfeszültség

lemezek/tárcsák
%----------------------------------------------------------------------------
\section{Jelek}
%----------------------------------------------------------------------------
feszültség
olvasás
feldolgozás

%----------------------------------------------------------------------------
\section{Szabályozás}
%----------------------------------------------------------------------------

%----------------------------------------------------------------------------
\subsection{Szabályozás eszközei}
%----------------------------------------------------------------------------

\subsubsection{Eszközválasztás}

feszültségkonverzió --> limitált eszközök
A \ref{table:progrele} táblázat tartalmazza a programozható relék közül a megfelelő feszültséggel operálókat.

\begin{table}[]
	\renewcommand{\arraystretch}{1.4}
	\scalebox{0.73}{
		\begin{tabular}{|l||l|l|l|l|}
			\hline
			& \begin{tabular}[c]{@{}l@{}}Schneider Electric\\ Zelio Logic\end{tabular} & IDEC SmartRelay    & Siemens LOGO! 							 & Eaton Easy  \\ \hhline{|=||=|=|=|=|}
			Modellszám           & SR2B121JD                                                                & FL1F-B12RCE                                                 & 6ED1052-2MD08-0BA2                                                     & EASY-E4-UC-12RCX1                                                                     \\ \hline
			Feszültség           & 12V                                                                      & 12/24V                                                      & 12/24V                                                                 & 12/24V                                                                                \\ \hline
			Bemenetek            & 4 DI + 4 AI                                                              & 4 DI + 2 AI                                                 & 4 DI + 4 DI/AI                                                         & 4 DI + 4 DI/AI                                                                        \\ \hline
			Kimenetek            & 4 DO                                                                     & 8 DO                                                        & 4 DO                                                                   & 4 DO                                                                                  \\ \hline
			Számítási frekvencia & 1 kHz                                                                    & 5 kHZ                                                       & 5 kHz                                                                  & 5 kHz                                                                                 \\ \hline
			Kommunikáció         & N/A                                                                      & N/A                                                         & Modbus TCP                                                             & MODBUS TCP/IP                                                                         \\ \hline
			Program felület      & \begin{tabular}[c]{@{}l@{}}ZelioSoft 2\\ (LD, FBD)\end{tabular}          & \begin{tabular}[c]{@{}l@{}}WingLGC\\ (LD, FBD)\end{tabular} & \begin{tabular}[c]{@{}l@{}}LOGO! Soft Comfort\\ (LD, FBD)\end{tabular} & \begin{tabular}[c]{@{}l@{}}EASYSOFT-SWLIC/easySoft7\\ (EDP, LD, FBD, ST)\end{tabular} \\ \hline
			Internet csatlakozó  & Nincs                                                                    & Ethernet RJ45                                               & Ethernet RJ45                                                          & Ethernet RJ45                                                                         \\ \hline
			SD memory            & Nincs                                                                    & MicroSD                                                     & MicroSD                                                                & Nincs                                                                                 \\ \hline
			Méret                & 90x68x10                                                                 & 90x71.5x58                                                  & 90x71.5x58                                                             & 90x72x58                                                                              \\ \hline
			Költség              & 75 000                                                                   & 69 000                                                      & 58 000                                                                 & 72 000                                                                                \\ \hline
		\end{tabular}
	}
	\caption{Programozható relék összehasonlítása}
	\label{table:progrele}
	\renewcommand{\arraystretch}{1}
\end{table}


%----------------------------------------------------------------------------
\subsection{Programozás}
%----------------------------------------------------------------------------
milyen módszerrel
ki tudja felprogramozni
kapcsolódás
%----------------------------------------------------------------------------
\subsection{Adatfeldolgozás}
%----------------------------------------------------------------------------
analóg feszültségjelek
fordulatszám kialakító függvények
kiadott jelek - hiba felismerése
%----------------------------------------------------------------------------
\subsection{Hibatűrő rendszer kialakítása}
%----------------------------------------------------------------------------
várakozás a megcsúszással
%----------------------------------------------------------------------------
\subsection{Szennyeződés kizárása}
%----------------------------------------------------------------------------
doboz, ip védettség
tömszelence
%----------------------------------------------------------------------------
\section{Visszajelzés}
%----------------------------------------------------------------------------
mikor kell jelezni, mit érzékelnek
%----------------------------------------------------------------------------
\subsection{Visszajelzés eszközei}
%----------------------------------------------------------------------------
hang haszontalan
fény láthatóság
%----------------------------------------------------------------------------
\subsection{Kommunikáció}
%----------------------------------------------------------------------------
digitális kimenet

\subsubsection{Opcionális hálózati csatlakozás}

\subsubsection{Gyártói szabványos kommunikáció}