%----------------------------------------------------------------------------
\chapter*{\eloszo}\addcontentsline{toc}{chapter}{\eloszo}
%----------------------------------------------------------------------------

Ez a szakdolgozat a Budapesti Műszaki- és Gazdaságtudományi Egyetem gépészmérnöki karán, mechatronikai mérnöki alapszakon készült. A dolgozat a szerző mezőgazdasági automatizáció felé való érdeklődése, valamint a Hevesgép Kft. munkatársainak felajánlása nyomán jött létre. A munka a "Szálastakarmány felszedő adapter szenzortechnikai fejlesztése" címet kapta, melynek pontos témáját a felek közös megegyezés alapján a cég automatizálási projektjei közül választották.

\begin{center}
    $\thicksim \; \thicksim \; \thicksim$
\end{center}


\subsubsection*{Köszönetnyilvánítás}
\emph{Szeretném megköszönni a Hevesgép Kft. ügyvezető igazgatójának, Vincze Bálintnak és műszaki igazgatójának, Lányi Róbertnek a lehetőséget, hogy ezt a dolgozatot elkészíthessem, valamint a sok segítséget, amit tőlük kaptam, Haba Tamásnak, a MOGI tanszék doktoranduszának és témavezetőmnek a figyelmes és gyors segítséget a dolgozat írása közben, és Dr. Szabó Tibornak hogy fordulhattam hozzá kérdéseimmel.}
Ez a sablon a Villamosmérnöki és Informatikai Kar Méréstechnika és Információs Rendszerek Tanszék szakdolgozat és diplomaterv sablonja alapján készült. Köszönöm készítőinek és karbantartóinak a munkájukat.


\vspace{0.5cm}

\begin{flushleft}
{Budapest, \today}
\end{flushleft}

\begin{flushright}
\emph{\authorName}
\end{flushright}

\vfill
