%----------------------------------------------------------------------------
\chapter{Szakirodalmi áttekintés}
\label{sec:Szakirodalom}
%----------------------------------------------------------------------------
\section{Szenzorok fajtái}
%----------------------------------------------------------------------------

\subsection{Mérendő mennyiségek}

A feladatom során, a nyomatékhatároló csúszásának meghatározásához az azt megelőző és azutáni tengelyek fordulatszámának összehasonlítására van szükség. Egy tengely fordulatszámának mérésére több megközelítés is létezik. Mérhetjük közvetlenül a tengely elfordulását, akár egy fordulatszámonként egyszer történő jeladást, vagy érzékelhetünk folyamatos változást a tengely kerületén. A jeleket akár integrálhatjuk gyorsulásmérésből, vagy deriválhatjuk szögelfordulásból, azonban ezeknek a pontossága nem minden esetben megfelelő, valamint a számítási igénye is magasabb az ilyen módon nyert jeleknek.

\subsection{Mérési elvek}

\subsection{Szenzor kialakítások}
%----------------------------------------------------------------------------
\section{Jelek feldolgozásának menete}
%----------------------------------------------------------------------------

%----------------------------------------------------------------------------
\section{Szabályozás módszerei}
%----------------------------------------------------------------------------

%----------------------------------------------------------------------------
\section{Visszajelzés lehetőségei}
%----------------------------------------------------------------------------