%----------------------------------------------------------------------------
\chapter{Szakirodalmi áttekintés}
\label{sec:Szakirodalom}
%----------------------------------------------------------------------------
\section{Szenzorok fajtái}
%----------------------------------------------------------------------------

A szenzorválasztás az alapja a folyamatnak. Meghatározza milyen technológiákkal, módszerekkel hajtjuk végre a méréseket, és ezáltal milyen kapacitású, programozású rendszert kell válasszunk. A szenzorokat érintő kutatómunka ezt a választási folyamatot segíti elő, hiszen az összes népszerű lehetőség tudatában lehet egy informált döntést hozni.

\subsection{Mérendő mennyiségek}

A feladatom során, a nyomatékhatároló csúszásának meghatározásához az azt megelőző és azutáni tengelyek fordulatszámának összehasonlítására van szükség. A harmadik fordulatszámmérés a felszedő tengelyen történik meg, kifejezetten az operátor informálása céljából. 

\subsection{Mérési elvek}

Egy tengely fordulatszámának mérésére több megközelítés is létezik. Lehetséges a tengely elfordulásának közvetlen mérése, akár fordulatonként egyszer történő jeladás regisztrálása, vagy a tengely kerületén érzékelhető folyamatos változás. A fordulatszám más mért mennyiségekből is származtatható, például integrálás útján gyorsulásmérésből, vagy deriválással szögelfordulásból, azonban ezeknek a pontossága nem minden esetben megfelelő, valamint a számítási igénye is magasabb az ilyen módon származtatott jeleknek.\\

\subsection{Elmozdulás érzékelése}

Az elmozdulás mérése általában egy adott távolság, szögelfordulás tartományán belül alkalmazható, így folyamatos elfordulást mérni csak limitált fordulatszám mellett alkalmasak. A fordulatok mérésénél ezért általában a kerület mentén történő távolságbeli különbség mérése valósul meg. Erre a kerület menti geometriát (pl. fogaskerekek), egy segédlemezzel kialakított változást, vagy akár egy felhelyezett jeladó (pl. mágnes) érzékelés is mérhető.

\subsubsection{Potenciométeres elmozdulásmérés}

Az alapvető potenciométer egy lineáris vagy elfordulásmérésre alkalmas szenzor. Az elfordulást lehetséges 0-360$\degree$ között mérni egy kör kerületén, vagy akár annak egy részletén is, azonban ha 360$\degree$ fölötti mérésre van szükség egy spirális szalagot szokás alkalmazni, amelyek akár 60 fordulatot is képesek mérni. \cite{Morris2016} 

\subsection{Szenzor kialakítások}

%----------------------------------------------------------------------------
\section{Jelek feldolgozásának menete}
%----------------------------------------------------------------------------

%----------------------------------------------------------------------------
\section{Visszajelzés lehetőségei}
%----------------------------------------------------------------------------

%----------------------------------------------------------------------------
\section{Szabályozás módszerei}
%----------------------------------------------------------------------------
