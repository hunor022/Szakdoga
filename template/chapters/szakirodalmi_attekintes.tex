%----------------------------------------------------------------------------
\chapter{Szakirodalmi áttekintés}
\label{sec:Szakirodalom}
%----------------------------------------------------------------------------
\section{Szenzorok fajtái}
%----------------------------------------------------------------------------

\subsection{Mérendő mennyiségek}

A dolgozatom során a feladat szerint több fordulatszám mérés összehasonlításával foglalkoztam. 
Egy tengely fordulatszámának mérésére több megközelítés is rendelkezésünkre áll, mérhetjük közvetlenül a tengely elfordulását, akár egy fordulatszámonként egyszer történő jeladást, vagy érzékelhetünk folyamatos változást a tengely kerületén. A jeleket integrálhatjuk gyorsulásból, vagy deriválhatjuk szögelfordulásból, azonban ezeknek a pontosságához nagy számítási igényre van szükségünk.

\subsection{Mérési elvek}

\subsection{Szenzor kialakítások}
%----------------------------------------------------------------------------
\section{Jelek feldolgozásának menete}
%----------------------------------------------------------------------------

%----------------------------------------------------------------------------
\section{Szabályozás módszerei}
%----------------------------------------------------------------------------

%----------------------------------------------------------------------------
\section{Visszajelzés lehetőségei}
%----------------------------------------------------------------------------