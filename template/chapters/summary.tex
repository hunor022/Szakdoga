%----------------------------------------------------------------------------
\chapter{\osszefoglalas} % (Eredmények értékelése)
%----------------------------------------------------------------------------

%----------------------------------------------------------------------------
\section{Alkalmazási lehetőségek}

%----------------------------------------------------------------------------
\subsection{Feladat kivitelezésének lehetőségei}
%----------------------------------------------------------------------------

%----------------------------------------------------------------------------
\subsection{Üzembiztonsági megoldások}
%----------------------------------------------------------------------------

%----------------------------------------------------------------------------
\subsection{Diagnosztikai feladatok kivitelezése}
%----------------------------------------------------------------------------

%\section{Eredmények}
%Az összefoglaló értékelés a három oldalt lehetőleg ne haladja meg! 
%Az elvégzett munka és eredményeinek bemutatása egyes szám első személyben fogalmazva.
%
%
%\section{Javaslatok/Következtetések/Tanulságok} % Válassz egyet
%A feladat elkészítése során levont tanulságok összefoglalása. Javaslattétel, 
%továbbfejlesztési lehetősége bemutatása, előretekintés a jövőbe stb.

% Keltezés, aláírás
\vspace{0.5cm}

\begin{flushleft}
{Budapest, \today}
\end{flushleft}

\begin{flushright}
\emph{\authorName}
\end{flushright}

\vfill
