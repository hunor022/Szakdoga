\documentclass{article}

\usepackage[english]{babel}

\usepackage[letterpaper,top=2cm,bottom=2cm,left=3cm,right=3cm,marginparwidth=1.75cm]{geometry}
\usepackage{parskip}
\usepackage{amsmath}
\usepackage{graphicx, animate}
\graphicspath{ {./images/} }
\usepackage[colorlinks=true, allcolors=blue]{hyperref}

\title{Jegyzetek}
\author{Kovács Hunor}

\begin{document}
	\maketitle
	
	\section{Feladatkiírás}
	
	\begin{enumerate}
		
		\item Tárja fel a mezőgazdasági gépeken használatos szenzortechnikai, adatfeldolgozási és visszajelzési megoldásokat!
		\item Válasszon a szálastakarmány felszedő adapteren alkalmazható fordulatszám meghatározására alkalmazható szenzort!
		\item Tervezzen az adapterhez és a szenzorhoz megfelelő adatfeldolgozó és visszajelző rendszert!
		\item Vizsgálja meg a tervezett mérőrendszer alkalmazási lehetőségeit üzembiztonsági és diagnosztikai feladatok esetén!
		\item Foglalja össze a kapott eredményeket magyar és angol nyelven.
	\end{enumerate}
	
	\section{Kutatómunka}
	\subsection{kiindulás}
	\begin{itemize}
		\item forgó mozgás milyen szenzorokkal mérhető?
		\item milyen fizikai mennyiséget lehet mérni? (szögpozíció, szögsebesség)
		\item milyen alapvető eszköztípusok vannak erre? (pl. enkóderek, tachogenerátor)
		\item milyen fizikai elven működnek ezek az eszközök? (optikai, kapacitív, mágneses, induktív stb.)
		\item a szenzorok milyen villamos jelet állítanak elő? milyen ipari szabványok vannak erre?
		konkrét, kereskedelmi forgalomban kapható eszközök kivitele (rögzítés iránya, tengelyméretek)
	\end{itemize}
	
	\subsection{később fontos lehet}
	\begin{itemize}
		\item milyen eszközzel digitalizáljuk majd a szenzorok jeleit? (PLC, mikrovezérlő, ipari PC stb)
		\item milyen módon továbbítsuk ezeket a mért adatokat? (buszok, protokollok, pl. soros port, CAN busz stb.)
	\end{itemize}
	
	\subsection{Paraméterek}
	\begin{itemize}
		\item Feszültség
		\item Fogaskerekek anyaga
		\item Fordulatszám intervallum
		
	\end{itemize}
	
	
	
	\section{Forgó mozgás szenzorai}
	
	\subsection{Handbook of Modern Sensors: Physics, Designs, and Applications}
	\subsubsection{Displacement/position sensors}
	Design guidelines:
	\begin{enumerate}
		\item How big is the displacement and of what type (linear, circular)?
		\item What resolution and accuracy are required?
		\item What the measured (moving) object is made of (metal, plastic, fluid, ferromagnetic, etc.)?
		\item How much space is available for mounting the detector?
		\item What are the environmental conditions (humidity, temperature, sources of
		interference, vibration, corrosive materials, etc.)?
		\item How much power is available for the sensor?
		\item How much mechanical wear can be expected over the life time of the machine?
		\item 8. What is the production quantity of the sensing assembly (limited number,
		medium volume, mass production)?
		\item 9. What is the target cost of the detecting assembly?
	\end{enumerate}
	
	Potenciométeres elfordulás mérés - nem alkalmas több fordulat megtételére kialakítása miatt
	Kapacitív szenzorok - szennyeződések miatt nem alkalmas
	Induktív/Mágneses szenzorok
	\begin{itemize}
		\item LVDT/RVDT - nem jo, kontakt alapu
		\item Örvényáram szenzor (Eddy current) - vastag testek detektálása (detektor távolságnál vastagabb kell legyen a test), por/olaj álló, felületi hibák, vastagságok mérésére alkalmas (overkill probs)
		\item Transverse Inductive Sensor - kis elmozdulás mérése (overkill)
		\item Hall effektus szenzorok - egy bizonyos távolságon (release point) belül lévő mágneseket érzékelnek
		\item Magnetoresistive sensors - mágneses forrás szükséges, nem mindegy milyen tengelyei hatnak a szenzorra, nagy pontosságú párhuzamosság kell hozzá (nem valószínű hogy jó ide)
		\begin{figure}[h]
			\caption{fogaskerék mérő elrendezés}
			\centering
			\includegraphics[width=0.7\textwidth]{magnetoresistive_toothed_wheel}
		\end{figure}
		\item Magnetostrictive detector - cső és gyűrűmágnes kell hozzá (nem jó)
	\end{itemize}
	Optikai szenzorok - olaj és porszennyezés miatt nem is jön szóba
	Radarok, pointing deviceok
	
	
	\subsubsection {Velocity and Acceleration}
	Konkretan csak a gyorsulasrol volt szo
	
	\subsection{Alan S Morris, Reza Langari Ph.D. - Measurement and Instrumentation}
	
	\subsubsection{Rotational Motion Transducers / Rotational Velocity}
	
	Digital Tachometers - a tengely körüli jelöléseket érzékeljük, a sűrűségük és a szenzor órajele határozza meg a felbontást és az érzékenységet
	
	\begin{itemize}
		\item Optical - a tárcsán/tengelyen visszaverő sávok vagy lyukak (egyik oldalon emitter másikon detektor) - érzékeny por és olaj szennyezésre
		\item variable reluctance sensors
		\begin{itemize}
			\item ferromágneses anyagok (vas) érzékelése
			\item fogaskeréken a távolodó és közeledő fogak megváltoztatják a mágneses fluxust amit mér a szenzor
			\item van egy fordulatszám limit (?? - jóval kevesebb mint 10 000 rpm)
		\end{itemize}
		\begin{figure}[h]
			\caption{fogaskerék mérő elrendezés}
			\centering
			\includegraphics[width=0.7\textwidth]{images/variable reluctance sensor.png}
		\end{figure}
		\item Hall effektus szenzor:  fogaskerekeknél a szenzor mögött egy állandó mágneses tér nagyobb amikor a fogak közötti hely van a szenzor előtt, kisebb a fogak eltérítik a mágneses tér egy részét, a szenzor ez alapján ad le különböző feszültségeket
	\end{itemize}
	Stroboszkópos szenzorok - fény alapúak, nem alkalmasak \\
	analóg mérők - kevésbé pontosak\\
	be kell építeni\\
	hasonló a helyzet a giroszkópok, optikus szál giroszkópok, és mems (mikroelektromechanikai rendszerű) giroszkópok esetén\\
	\\
	pozíció / gyorsulás általi számolás pontatlan\\
	\\
	kalibráció: stroboszkópos fordulatszám mérővel, 0.1\%-nál kisebb a hiba\\
	
	\section{Beszerzés}
	
	\subsection{meeting 10.8.}
	
	12V a rendszer
	szenzor választás - komplett házas - szennyeződés
	dugalyos cserélhető verzió - kábel megválasztása
	84391778 H4CQF
	
	M12-es: \url{https://www.electricstore.hu/autonics/prdcm12-4dp-induktiv-erzekelo-12-24vdc-m12-4mm-pnp-no?gad_source=1&gclid=CjwKCAjw3624BhBAEiwAkxgTOkLaKXLI6GQJ9YZEVY_-tAHNSpkxoH8HQei1cQJeC9ptuPmOc45EvBoCwnoQAvD_BwE}
	M30-as: \url{https://www.ipari-elektronika.com/sick-induktiv-erzekelo-ime30-10bpszc0s-1040998?utm_source=google_shopping&utm_medium=cpp&utm_campaign=direct_link&utm_term=&utm_campaign=&utm_source=adwords&utm_medium=ppc&hsa_acc=7898010743&hsa_cam=20569131558&hsa_grp=&hsa_ad=&hsa_src=x&hsa_tgt=&hsa_kw=&hsa_mt=&hsa_net=adwords&hsa_ver=3&gad_source=1&gclid=CjwKCAjw3624BhBAEiwAkxgTOmBXvgHD_t8CGmibL7POGAecy22Od8zOcuYa143QJrFU7jOKfMmOjRoCRAoQAvD_BwE}
	
	
	3 szenzor mindenképp
	fogastárcsa a felszedőhöz (85-150 rpm)
	túloldal a csiga tengelyen (220-270 rpm) - lemez kell
	nyomatékhatároló külső fogastárcsa
	
	fogas lemez - 1 füles vagy sok füles melyik pontosabb
	
	2 féle nyomatékhatároló - egyikre megoldás
	
	
	
	
\end{document}
